A device that helps perceive your surrounding by using the sense of touch with deployment of Y\+O\+L\+Ov5 through Lib\+Torch C++ A\+PI.



   \href{https://github.com/Haptic-Vision/haptic_vision/blob/main/LICENSE}{\tt } ~\newline
 \href{https://github.com/MataPOS/matapos/issues}{\tt Report Bug} \href{https://github.com/MataPOS/matapos/issues}{\tt Request Feature} ~\newline


 

\section*{Requirements}


\begin{DoxyItemize}
\item \href{https://www.instructables.com/Install-Ubuntu-18044-LTS-on-Your-Raspberry-Pi-Boar/}{\tt Ubuntu 18.\+04}
\item Open\+CV 3.\+2.\+0
\item \href{https://download.pytorch.org/libtorch/nightly/cpu/libtorch-shared-with-deps-latest.zip}{\tt Lib\+Torch 1.\+6.\+0}
\item \href{https://askubuntu.com/questions/355565/how-do-i-install-the-latest-version-of-cmake-from-the-command-line}{\tt C\+Make 3.\+10.\+2}
\item \href{https://abyz.me.uk/rpi/pigpio/download.html}{\tt pi\+G\+P\+IO}
\end{DoxyItemize}

\section*{Software-\/stack}

To setup and run the device use the following steps\+:



\subsection*{To run}


\begin{DoxyEnumerate}
\item Update raspberry pi 
\begin{DoxyCode}
sudo apt update
sudo apt upgrade
\end{DoxyCode}

\item Install pigpio 
\begin{DoxyCode}
sudo apt-get install libpigpio-dev
\end{DoxyCode}

\item Install Qt 
\begin{DoxyCode}
sudo apt-get install qtdeclarative5-dev-tools
sudo apt-get install libqwt-qt5-dev
\end{DoxyCode}

\item Install Open\+CV 
\begin{DoxyCode}
sudo apt install libopencv-dev
sudo apt install libopencv-core-dev
\end{DoxyCode}

\item Setup Pi Cam Once the ribbon cable is connected to the rasperry pi. Got to raspberry pi configurator 
\begin{DoxyCode}
sudo raspi-config
\end{DoxyCode}

\item Go to interface and enable camera option. Then restart you Pi.
\item Once Pi is restarted, check of camera is working by 
\begin{DoxyCode}
libcamera-jpeg -o image.jpg
\end{DoxyCode}

\item If raspicam dependencies are not installed, follow steps in the following \href{https://github.com/cedricve/raspicam}{\tt link}.
\item Clone program files onto the rasperry pi.
\item Compile and run. 
\begin{DoxyCode}
mkdir build && cd build
cmake ..
make
./../bin/HViz\_run
\end{DoxyCode}

\end{DoxyEnumerate}

Note\+:
\begin{DoxyItemize}
\item Libtorch package solely contributes to the presence of python files in the project.
\item C\+O\+C\+O-\/pretrained Y\+O\+L\+Ov5s model has been provided. For more pretrained models, see \href{https://github.com/ultralytics/yolov5}{\tt yolov5}.
\end{DoxyItemize}

\section*{Circuit Design}



\section*{Technologies}

H\+Viz is built using\+:


\begin{DoxyItemize}
\item \href{https://www.cplusplus.com/}{\tt C++ Programming Language}
\item \href{https://www.linux.org/}{\tt Debian/\+Ubuntu Linux}
\item \href{https://www.raspberrypi.org}{\tt Raspberry Pi}
\item \href{https://cmake.org/}{\tt Cmake}
\item \href{https://opencv.org/}{\tt Open\+CV}
\item \href{https://github.com/google/googletest}{\tt Google Test}
\item \href{https://www.doxygen.nl/index.html}{\tt Doxygen}
\item \href{https://www.qt.io/}{\tt Qt}
\end{DoxyItemize}

\subsection*{Socials}

\href{https://www.instagram.com/hapticvision_/}{\tt }


\begin{DoxyItemize}
\item \href{https://www.instagram.com/hapticvision_/}{\tt Instagram}
\end{DoxyItemize}

\subsection*{The Team}


\begin{DoxyItemize}
\item \href{https://github.com/rdj2829}{\tt }
\item \href{https://github.com/dheerajsankar}{\tt }
\item \href{https://github.com/kprakz}{\tt }
\item \href{https://github.com/josephjoel3099}{\tt }
\end{DoxyItemize}

\subsection*{Contact Us}


\begin{DoxyItemize}
\item DM us on \href{https://www.instagram.com/hapticvision_/}{\tt }
\item Email us at $\ast$$\ast$hapticvisionuofg.com$\ast$$\ast$
\end{DoxyItemize}

\subsection*{License}